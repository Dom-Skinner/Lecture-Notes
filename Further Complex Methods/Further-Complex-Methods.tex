\documentclass{article}
\usepackage{amsmath}
\usepackage{bm}
\usepackage{framed}
\usepackage{enumerate}
\usepackage{array}
\usepackage{amsfonts}
\usepackage[margin=3cm]{geometry}

\newcommand{\definition}{\textbf{Definition:}} 
\newcommand{\example}{\textbf{Example:}} 
\newcommand{\examples}{\textbf{Examples:}} 
\newcommand{\question}{\textbf{Question:}} 
\newcommand{\theorem}{\textbf{Theorem:}} 
\newcommand{\Log}{\mathrm{Log} \,} 
\newcommand{\improp}{\int_{-\infty}^{\infty} } % For improper integrals, standard limits

\begin{document}

\title{Further Complex Methods}
\author{Course given by Prof. M.Perry \\
\LaTeX\  by Dominic Skinner \\
Dom-Skinner@github.com}
\maketitle
\begin{tabular}{lr}
\textbf{Books:} & ``Complex Variables,'' M.J Ablowitz \& A. Fokes (CUP)\\
& ``A Course in Modern Analysis,'' Whittaker \& Watson
\end{tabular}
\\
\section*{Introduction}
Much of this section will be a recap of things learnt in the IB courses
Complex Methods/Complex Analysis. In particular, the first three lectures
seem to cover material familiar to anyone who understood IB Complex Methods.
\\

Any function of $x,y$ can be written as a function of $z , \;\bar{z}$ for
$z = x+iy$.
Functions of a complex variable are defined to be those functions of $x$ and
$y$ that can be written entirely in terms of $z$ only.
A function of a complex variable is continous if
\[ \lim_{z \to z_0} f(z) = f(z_0) \quad \mbox{(as in real analysis) } \]
The derivative of a function of a complex variable is
\[ f'(z) - \lim_{\delta z \to 0} \frac{f(z+ \delta z) - f(z)}{\delta z} \]
For a function to be differentiable, the limit must be independent of
the direction that the limit is taken.
If this is true, then the function is said to be differentiable at $z$.
If $f'(z)$ exists, then $f(z)$ is continous (converse not true).
%
%
\subsubsection*{Cauchy Riemann equations}
Write $f(z) = u(x,y) + iv(x,y)$ with $u,v$ both real. Then
\[ dz \, f'(z) = \lim_{\delta z = \delta x + i \delta y \to 0} \left(
u(x+\delta x, y+ \delta y) + iv(x+\delta x,y+\delta y) - u(x,y) - iv(x,y) 
\right) \]
If $\delta y = 0$, $dz = dx$ and we get that
\[ f'(z) = u_x + i v_x \]
Suppose now that $\delta x = 0$.
\[ i \delta y\, f'(z) = u_y+iv_y \]
\[ \implies f'(z) = v_y - i u_y \]
\[ \implies v_y - iu_y = u_x + iv_x \]
\begin{framed}
\[ \implies \left. \begin{array}{cr}
v_y =& u_x \\
v_x =& -u_y \end{array} \right\} \mbox{ The Cauchy-Riemann equations} \]
\end{framed}
If the Cauchy-Riemann equations (C-R) hold, the derivatives exist and are 
continous, then $f(z)$ is differentiable. If the C-R equations hold
then $u, \; v$ are harmonic.
\[ u_{xx} = v_{xy} = -u_{yy} \implies u_{xx}+u_{yy} =0 \]
A similar equation holds with v.
\\
% A diagram goes here
\\
Consider surfaces of $u=const$, $v=const$. These surfaces are orthogonal.
\begin{align*}
\nabla u &= (u_x,u_y) \mbox{ - normal to } u = const \\
\nabla v &= (v_x,v_y) \mbox{ - normal to } v = const 
\end{align*}
and so
\[ \nabla u \cdot \nabla v = u_xv_x + u_yv_y =0 \mbox{ from C-R} \]
%
%
\subsubsection*{Analytic functions}
\definition\ Analytic function
\\
$f(z)$ is analytic at $z_0$ if $f(z)$ is differentiable in some neighbourgood
of $z_0$. $f(z)$ is analytic in a region if a similar condition applies.
\\
\\
\examples\
\begin{enumerate}[(i)]
\item $e^z$  is analytic in the finite complex $z$-plane
\item $\bar{z}$ is analytic nowhere
\item $1/z^3$ is analytic everywhere except at $z=0$
\end{enumerate}
\noindent \definition\ Entire functions
\\
A function is entire if it is analytic in the finite complex plane
\\
\\
\examples\
\begin{enumerate}[(i)]
\item $e^z$ , this only fails to be analytic at $\infty$ 
\item $\sin z$ 
\item $z^2$ 
\end{enumerate}
~\\
\definition\ Isolated singularity
\\
A function is said to have an isolated singularity if it fails to
be analytic at a point.
\\
\\
\example\ $1/z^3$ has an isolated singularity at the origin.
\\
\begin{framed}
\noindent Suppose that a function has an isolated singularity at $z=z_0$. Then
it can be expanded as a Laurent series around $z_0$.
\[ f(z) = \sum_{-\infty}^{\infty} c_n (z - z_0)^n \]
Note that this sum is over all positive and negative powers.
\end{framed}
\noindent
Suppose that $c_n=0$ for all $n< -N$ where $N>0$. 
\begin{itemize}
\item If $c_n = 0 \; \forall \, n >0$ then it is not singular.
\item If $c_n = 0$ for all $n<-N$ for $N>0$, then one has a pole of order
	$N$.
\end{itemize}
\noindent \example\ $1/z^3$ has a pole of order 3 at $z=0$.
\\
The coefficient $c_{-1}$ is special, it is the residue of the pole at 
$z_0$.
\\
\\
\definition\ Removable singularities
\\
Fake singularities where the building blocks of $f(z)$ have isolated
singularities, but $f(z)$ does not.
\\
\\
\example\
\[ f(z) = \frac{\sin z}{z} = \frac{1}{z} ( z - \frac{z^3}{6} + \dots) 
= 1 - \frac{z^2}{6} \]
$f(z)$ has a removable singularity at $z=0$.
\\
\\
\example\
\[ f(z) = \frac{1}{z} - \frac{1}{z + z^2} = \frac{1}{1+z} \]
so $f(z)$ has a removable singularity at the origin.
\\
\\
\definition\ Essential Singularity \\
An essential singularity is where the order of the pole of an isolated
singularity is infinite.
\\
\\
\example\ $f(z) = e^{1/z}$, $z=0$ is an isolated singularity, as a Laurent
series
\[ f(z) = \sum_{\infty}^{0} \frac{1}{(-n)!} z^n \]
Note that in this exaple, $f(z)$ is not even continous at $z=0$, its value
depends on how one approaches $z=0$.
\\
\\
\definition\ Meromorphic functions
\\
These are functions of $z$ that only have poles of any finite order
in the finite complex plane. \\
\\
\examples\
\begin{enumerate}[(a)]
\item $1/z^2$ has a pole of order 2 at the origin.
\item $\cot z$ has poles of order 1 at $z = n\pi, \, n \in \mathbb{Z}$
\end{enumerate}
\noindent \theorem\ Cauchy's Theorem
\[ \int_C f(z) \, dz = 2 \pi i \left( \sum \mbox{ Residues of the poles
enclosed by C} \right) \]
The integral is taken around $C$ in the anti-clockwise direction, and
$f(z)$ is meromorphic.
\\
\subsubsection*{The Riemann Sphere}
The complex plane is really a sphere, the Riemann sphere.
% Big Diagram of the Rieman sphere goes here
\\
$w$ is the perpendicular distance from the $x-y$ plane.
The north pole corresponds to infinity, and all of infinity 
has become a point.
\\
\\
\definition\ Stereographic projection
\\
Construct a straight line staring at $N$, through $P$ to meet the
complex plane at $C$.
\[ N = (0,0,2), \qquad P = (X,Y,\,) \]
Construct this line by saying that $s$ is a parameter along the line 
such that $s=0$ at the north pole and $s=1$ at $P$.
\[ \left. \begin{array}{cc}
x &= X_s \\
y &= Y_s \end{array} \right\} \mbox{ What about w?} \]
\[ w = 2 - (1 \pm \sqrt{1-X^2-Y^2})s \]
at $C$, $w=0$
\[ \implies s = \frac{2}{1 \pm \sqrt{1 - X^2 -Y^2}} \mbox{ at } C \]
Hence the coordinates of the point $C$ are
\[ x= \frac{2X}{1 \pm \sqrt{1 - X^2 -Y^2}} \;, \quad
 y= \frac{2Y}{1 \pm \sqrt{1 - X^2 -Y^2}} \]
Thus if $X,Y$ both $\to 0$, then $x,y \to \infty$ with the choice of sign.
All of infinity gets mapped to the north pole of the Riemann sphere. This
motivates how to think about infinity.
\[ z \mapsto 1/z = w \mbox{ maps infinity to the origin} \]
Suppose $f(z)=z$, then $f(w) = 1/w$.
$f(w)$ has a simple pole of residue 1 at $w = 0$
$\implies f(z) = z$ has a simple pole of residue 1 at infinity.
\\
\\
This holds true for any function of a complex variable; to 
examine the behaviour of a funcion at $\infty$, send 
$z \mapsto 1/z = w$ and ask what happens at $w=0$.
\\
\\
\example\ $f(z) = e^z= e^{1/w}$ has an essential singularity at $w=0$.
% a diagram of the essential singularity.
\\
\\
\theorem\ Liouville's theorem
\\
If $f$ is analytic everywhere including $\infty$ then it must be a constant.
\\
\\
\subsection*{Multi-valued functions}
For a real variable, the square root of a positive number has two forms
$\pm \sqrt{x} $.
\\
Now consider $z^{1/2}$ and decompose into modulus and argument.
\[z^{1/2} = \rho^{1/2} e^{i\theta /2} \]
As one moves around the circle,
\begin{align*}
\theta & \mapsto \theta + 2\pi \\
z^{1/2} & \mapsto \rho^{1/2} e^{i(\theta + 2\pi)/2} = -\rho^{1/2}
e^{i\theta /2} 
\end{align*}
So $f(z)$ changes sign. If one goes around the circle twice then
$\theta \mapsto \theta + 4\pi$, and so $f(z)$ is invariant.
\\
\begin{framed}
\noindent The effect of going around the circle is usually called the monodromy,
and for the case $f(z) = z^{1/2}$ this is $(-1)^n$.
\\
\\
The monodromy always forms a group. So in this case, the monodromy group
is just $\mathbb{Z}_2$,
\\
\\
A point where the monodromy is not 1 is called a branch point
\end{framed}
% Include diagram of branch point
For $f(z) = z^{1/2}$, the origin is a branch point. Around $z_0$
the function returns to its starting point. This holds for any $z_0 \neq 0$
in the finite complex plane.
Infinity is also a branch point.
\[ z \mapsto 1/w \, , \;\; w^{-\frac{1}{2}} \mapsto (\rho')^{-\frac{1}{2}} 
e^{-\frac{i\theta'}{2}}\]
and so when $\theta \mapsto \theta + 2\pi$, $w^{-1/2}$ changes sign.
Therefore $z= \infty, \; w=0$ is also a branch point.
There is always more than one branch point, so always look at infinity.
Branch points represent a failure of analycity.
\\
\\
\example\ $f(z)-(z-z_0)^p$. If $p$ is an integer, then $(z-z_0)^p$ is single
valued. Consider instead $p=m/n$ for $m,n$ integers.
%Include diagram$
\[ z = z_0 + \rho e^{i\theta}\]
\[ (z-z_0)^p = \rho^p e^{ip\theta} = \rho^{n/m}e^{im\theta /n} \]
Take $\theta \to \theta + 2\pi$. Then
\[ (z-z_0)^p \to \rho^{m/n} e^{2\pi i m / n} \]
The change of the phase of the function is $\displaystyle e^{2\pi im/n}$
for going round $z=z_0$ once anticlockwise.
\\

Suppose one goes around $z=z_0$, $s$ times, then the phase factor is 
$\displaystyle e^{2 \pi i m s /n}$. Thus if $s=n$ one gets back to the 
original value, or indeed if $s$ is any multiple of $n$ times.
The monodromy is therefore $\displaystyle e^{2\pi i m/n}, e^{4\pi i m/n},
\dots$. Thus the monodromy group is the cyclic group of order $n$, i.e.
$\mathbb{Z}_n$.
\\
\\
Suppose that $p$ is not rational. Then one never gets back to the starting
point. Monodromy for a single circle of $z_0$ is $\displaystyle e^{2\pi ip}$.
The monodromy group is $\mathbb{Z}_{\infty}$.
\\
\\
\example\ $f(z) = \Log z$. If $\displaystyle z = \rho e^{i\theta}$,
then set
\[f(z) = \log \rho + i\theta \]
So going around a circle around the origin once has the effect that
$\displaystyle \Log z \to \Log z + 2\pi i$.
If one goes around the circle $n$ times then 
$\displaystyle \Log z \to \Log z + 2n\pi i$.
Thus there are an infinite number of possible values for $\Log z$.
The monodromy is addition of $2\pi i$. The monodromy group is the 
integers under addition.
\\
\\
We can see that $z=0$ is a branch point, but
$z = \infty$ is also a branch point:
\[z \to 1/w, \quad \Log z \to \Log 1/w = - \Log w \]
Thus $w=0$ is also a branch point, and so $z=\infty$ is a branch point.
\\
\\
\example\ $f(z) = \sin ^{-1} z $ is multi-valued, since $\sin^{-1} z$ is
ambigous under the addition of $2\pi n$. 
\\
\subsection*{Branch Cuts}
This is a method of making $f(z)$ simgle-valued in the complex $z$ plane.
\\
\\
\example\ $f(z) = z^{1/2} $. For $z = \rho e^{i\theta}$ this is
\[f(z) = \underbrace{\rho^{1/2}}_{\geq 0} e^{i\theta/2} \]
$f(z)$ is single-valued if $\theta$ lies in a range of $2\pi$.
If we restrict $0 < \theta < 2\pi$ then the function is single valued, 
but discontinous across the positive real axis. This is a failure of 
analycity.
To get around this, exclude the positive real axis from the definition of the
function.
\\
\\
If $z$ is real, $z^{1/2}$ can still be defined either by
\begin{itemize}
\item taking the limit from the top half-plane
\item taking the limit from the bottom half-plane
\end{itemize}
There is always a discontinuity across a branch cut. The branch cut extends
all the way out to infinity since the discontinuity between $\theta =0$ and
$\theta = 2 \pi$ is non vanishing for all $\rho$.
\\
\\
For a square root type branch cut, the discontinuity is always just a sign 
corresponding to the nature of the monodromy. However, this is not the only
way to arrange a branch cut. Another possibility is for $\theta$ to run from
$-\pi$ to $+\pi$. In fact, one could (peversely) pick any $2\pi$ interval for
$\theta$ and it could be $\rho$ dependent.
% Figure of rho dependent branch cut
\\
\\
\example\ $f(z) = (z-1)^{1/4} $. There is a branch point at $z=1$. There are
four possible values for $f(z)$. If one goes around a little circle enclosing
$z=1$ then
\[ (z-1)^{1/4} \to e^{i\pi /2} (z-1)^{1/4}\] 
There is a branch point at infinity, send $z \to 1/w$,
\[ (z-1)^{1/4} \to \left( \frac{1}{w} -1 \right) ^{1/4} =
w^{-\frac{1}{4}} ( 1-w)^{\frac{1}{4}} \]
Hence a branch point at $w=0$ and so at $z=\infty$.
\\
\\
\example\ $\displaystyle I = \int^{\infty}_0 \frac{x^{1/2}}{1+x^2} dx$
with $x$ real. We can convert this to a contour integral taking
$\displaystyle I = \int^{\infty}_0 \frac{z^{1/2}}{1+z^2} dz$ where now
$\displaystyle f(z)= \frac{z^{1/2}}{1+z^2} $ has Branch points at
$0,\infty$ and simple poles at $\pm i$. We restrict the argument of $z$ to
run between $0$ and $2\pi$.
%Diagram of Keyhole integral
\[ \int_C \frac{z^{1/2}}{1+z^2} dz = 2\pi i \left( \mbox{residues at }
e^{i\pi /2} \mbox{ and } e^{-3i\pi /2} \right) \]
Along $C_1, \, z = x e^{0i}$ one just gets 
$\displaystyle I = \int^{\infty}_0 \frac{x^{1/2}}{1+x^2} dx$\\
Along $C_2, \, z = R e^{i\theta}$ for $R$ very large. Therefore
$dz = iR e^{i\theta} d\theta$ and
\[ \int_{C_2} f(z) dz = \int_0^{2\pi} \frac{R^{1/2} e^{i\theta/2}}{1 + R^2 e^{2i\theta}} 
i R e^{i\theta} d\theta = O(R^{-1/2}) \to 0 \mbox{ as } R \to \infty \]
Along $C_3, \, z = x e^{2\pi i}$  and so 
\[ \int_{C_3} f(z) dz = \int_{\infty}^0 \frac{x^{1/2} e^{\pi i} }{1+ x^2 e^{4\pi i}} dx
= \int_0^{\infty} \frac{x^{1/2}}{1+x^2} dx = I \]
Along $C_4, \, z = \epsilon e^{i\theta}$ for $\theta$ very small. Therefore
$dz = i\epsilon e^{i\theta} d\theta$ and
\[ \int_{C_4} f(z) dz = \int_{2\pi}^0 \frac{\epsilon^{1/2} e^{i\theta/2}}
{1 + \epsilon^2 e^{2i\theta}} i \epsilon e^{-i\theta} d\theta = O(\epsilon^{3/2})
\to 0 \mbox{ as } \epsilon \to 0 \]
What are the residues at $z = e^{i \pi /2}, \, e^{3 i \pi /2} $?
Since these are simple poles, it is easiest to find
\[ \lim_{z \to e^{i\pi/2}} \left( \frac{z - e^{i\pi /2}}{z^2 + 1} z^{1/2} 
\right) \to  \lim_{z \to e^{i\pi/2}} \left( \frac{1}{2z} z^{1/2} \right)
= \frac{1}{2} e^{-i\pi /4} \]
and at $z = e^{3i \pi /2}$ the residue is 
\[ \lim_{z \to e^{3i\pi/2}} \left( \frac{z - e^{3i\pi /2}}{z^2 + 1} z^{1/2} 
\right) \to  \lim_{z \to e^{3i\pi/2}} \left( \frac{1}{2z} z^{1/2} \right)
= \frac{1}{2} e^{-3i\pi /4} \]
\[ \implies \int_C \frac{z^{1/2}}{1+z^2} dz = 2\pi i \left(
\frac{1}{2} e^{-i \pi /4} + \frac{1}{2} e^{-3i\pi /4} \right) =
2\pi \sin \frac{\pi}{4} = \frac{\pi}{\sqrt{2}} \]
\\
\example\ Another example of a branch cut.
\[f(z) = \sqrt{(z-a)(z-b)}\]
with $a,b \in \mathbb{R}, \; \>0,\, b<0$ This has branch points at $z= a, \, b$.
What aboutat infinity? As usual take $z = 1/w$
\[f(w) = \sqrt{\left(\frac{1}{w} - a\right)\left(\frac{1}{w} -b\right)} 
= \frac{1}{w} \sqrt{(1-aw)(1-bw)}\]
which is fine apart from a pole at $\infty$, which is not the same as a branch 
point. 
% Insert diagram here
\[f(z) = \sqrt{(z-a)(z-b)} = \sqrt{\rho_1 \rho_2}e^{i(\theta_1+\theta_2)/2}\]
with say $0 \leq \theta_1 , \, \theta_2 < 2\pi$. In this case the cut is 
directly beween $a$ and $b$. Making a different choice, $-\pi \leq \theta_1 < \pi$.
and $0 \leq \theta_2 < \pi$.
% Diagram
\\
\\
This cut appears to end at $+\infty$ and $-\infty$ which is not a branch point.
However the complex plane is really a sphere, infinity is really a point and 
the cut just happens to go through the point $\infty$.
%
%
\subsection*{Cauchy Principal Value of an integral}
Sometimes it is possible to consruct the Cauchy Principal Value of an integral,
defined to be
\[ P \int_A^B f(x) dx = \lim_{\epsilon \to 0} \left[ \int^{x_0 - \epsilon}_A 
f(x) dx + \int^B_{x_0 + \epsilon} f(x) dx \right] \]
Where the letter P before an integral indicates that it is the Principal Value
(PV). Suppose that $f(x)$ has a singularity at $x_0$, then if the limit as
$\epsilon \to 0$ exists, then it defines the principal value.
\\
\\
If the integral were convergent, then the principal value coincides with the
original integral.
\\
\\
\examples\
\[ P\int_{-1}^2 \frac{dx}{x} = \lim_{\epsilon \to 0} \left[ \int_{-1}^{-\epsilon}
\frac{dx}{x} + \int^2_{\epsilon} \frac{dx}{x} \right] = 
 \lim_{\epsilon \to 0} \left[ 
\log |x| ^{-\epsilon}_{-1} + \log |x| ^2 _{\epsilon} \right] = \log 2 \]
Principal balues work nicely at poles. In the complex plane this turns out to be
rather convenient. Suppose one integrates a function that has a pole on the real 
axis.
% Diagram of contour with a green line
\\
\\
PV corresponds to the green contour.
\\
\\
This contour might be closed in the top half plane. Apply a small modification
of Cauchy's theorem:
\[ 2\pi i \left( \sum \mbox{ Residues in } C \right) = \int_{\Gamma} f(z) dz
+ \int_C f(z) dz + \int_{C'} f(z) dz \]
Where $\displaystyle \int_C f(z) dz$ corresponds to the principal value.
$C' = z_0 + \epsilon e^{i \theta} $ where $\theta$ runs from $\pi$ to $0$.
Near $z = z_0$,
\[f(z) = \sum_{n=-1}^{\infty} c_n ( z-z_0)^n \]
So this has residue $c_{-1}$ at $z = z_0$.
\[ \int_{\pi}^0 \sum_{n=-1}^{\infty} c_n ( z-z_0)^n  \, dz 
= \int_{\pi}^0 \sum_{n=-1}^{\infty} c_n \epsilon^n e^{in\theta} 
(i \epsilon e^{i\theta} d\theta) 
= \int_{\pi}^0 \sum_{n=-1}^{\infty} c_n \epsilon^{n+1} e^{i(n+1)\theta} 
i d\theta \]
In the limit $\epsilon \to 0$ the only remaining term is $n=-1$.
Therefore,
\begin{align*}
 = \int_{\pi}^0 c_{-1} i d\theta &= -i\pi c_{-1} \\
 \implies \int_{c'} f(z) dz &= -i\pi c_{-1} 
\end{align*}
This shows also that higher poles can give trouble. We will only look at the case
of simple poles in the integrand.
\\
\\
\example\ Calculate the PV of 
\[ P \int_{-\infty}^{\infty} \frac{e^{iax}}{x} dx \]
for $a$ real and positive
% Picture of generic contour
Apply our modified version of Cauchy's theorem to this contour.
\[ 2\pi i \left( \sum \mbox{ Residues in } C \right)  = 
\int_{|z| = R, \, Im(z) >0} \frac{e^{iaz}}{z}dz +
P \int_{-R}^{R} \frac{e^{iax}}{x} dx - i\pi c_{-1} \]
Where $c_{-1}$ is the residue of $\frac{e^{iax}}{x}$ at $x=0$.
\[ \implies 0 = 0 + P\int_{-\infty}^{\infty} \frac{e^{iax}}{x} dx - i \pi \]
\[ \implies P\int_{-\infty}^{\infty} \frac{e^{iax}}{x} dx = \pi i \]
Take the imaginary part of this expansion to get that
\[ P \int_{-\infty}^{\infty} \frac{\sin ax}{x} dx = \pi\]
Since the principal part of a convergent integral is the same
as the integral, one finds that
\[ \int_{-\infty}^{\infty} \frac{\sin ax}{x} dx = \pi \] and is indepenent
of $a$ for $a>0$.
\\
\\
\example\
\begin{equation}
I = \improp \frac{e^{px} - e^{qx}}{1 - e^x} dx \tag{$0 < p,q < 1$}
\end{equation}
This is singular at $x = 2 \pi n i$ where the denominator zero. The
singularity at zero is removable. Hence this integral is the same as its
principal value 
\[I = P\improp \frac{e^{px} - e^{qx}}{1 - e^x} dx  \]
we consider 
\[I = \improp \frac{e^{pz} - e^{qz}}{1 - e^z} dz  \]
Choose a rectangular contour
% Picture of a contour
\[ 0 = \underbrace{P\int^{R}_{-R} \frac{e^{pz} - e^{qz}}{1 - e^z} dz}_{C_1}
+ \underbrace{P \int^{-R+2\pi i}_{R+2\pi i} \frac{e^{pz} - e^{qz}}{1 - e^z} dz }_{C_4}
-\underbrace{i\pi \left( \mbox{Res at } 0 \right)}_{C_2}
-\underbrace{i\pi \left( \mbox{Res at } 2\pi i \right)}_{C_3} \]
Now we consider
\[ P \int \frac{e^{pz} - e^{qz}}{1-e^z} = P\int \frac{e^{pz}}{1-e^z} dz - 
P \int \frac{e^{pz}}{1-e^z} dz \]
The residue at 0
\[ \lim_{z \to 0} \frac{z e^{pz}}{1- e^z} = -1 \]
The residue at $2\pi i$
\[ \lim_{z \to 2 \pi i} \frac{(x - 2 \pi i)e^{px}}{1- e^x} = -e^{2 \pi i p} \]
\[\implies  0 = 
\int_C \frac{e^{pz}}{1-e^z} dz =
\underbrace{P\improp \frac{e^{px}}{1 - e^x} dx}_{I_1}
+ P \int^{-\infty}_{\infty} \frac{e^{p(x+2\pi i)}}{1 - e^{x+2\pi i}} dx 
-i\pi \left( \mbox{Res at } 0 \right)
-i\pi \left( \mbox{Res at } 2\pi i \right) \]
and so we can find $I_1$
\[ 0 = I_1 - I_1 e^{2\pi i p} - i\pi(-1) - i \pi ( -e^{2i\pi p}) \]
\[ \implies \frac{-i\pi (1+e^{2i\pi p})}{1 - e^{2\pi i p} }  =
\pi \cot \pi p\]
therefore
\[ I = \pi (\cot \pi p - \cot \pi q) \]
%
% 
\section*{The Hilbert Transform}
A Hilbert transform is somewhat like the Fourier transform. Take a function of 
a real variable $s$, $f(s)$. 
\begin{framed}
\noindent The Hilbert transform is 
\[ H_f(t) = \frac{1}{\pi} P \improp \frac{f(s)}{t-s} \]
where $t$ is another real variable.
\end{framed}
\noindent This can be turned into a form that is easier to work with.
\[ H_f(t) = \frac{1}{\pi} \lim_{\epsilon \to 0} \left[
\int_{-\infty}^{t-\epsilon}\frac{f(s)}{t-s} ds + \int^{\infty}_{t+\epsilon} 
\frac{f(s)}{t-s} ds \right] \]
Shift the pole in $s$ to the origin, $s-t = s'$
\[ H_f(t) = -\frac{1}{\pi} \lim_{\epsilon \to 0} \left[
\int_{-\infty}^{-\epsilon}\frac{f(s'+t)}{s'} ds' + \int^{\infty}_{\epsilon} 
\frac{f(s'+t)}{s'} ds' \right] \]
In the first integral take $s' \mapsto -s'$ to get that 
\[ H_f(t) = \frac{1}{\pi} \lim_{\epsilon \to 0} \left[
\int^{\infty}_{\epsilon} \frac{f(t-s')}{s'}
-\frac{f(t+s')}{s'} ds' \right] \]
Now lets find the inverse of the Hilbert transform. $\hat{H}_f(\omega)$ is
the fourier transform of the Hilbert transform. Here we consider the 
Fourier transform to be
\[ \hat{f}(\omega) = \improp f(t) e^{i\omega t} dt \]
Now
\[ \hat{H}_f(\omega) = \frac{1}{\pi} \lim_{\epsilon \to 0}
\improp \int_{\epsilon}^{\infty} (f(t-s)-f(t+s))e^{i\omega t} ds \, dt \]
We will assume that interchanging the orders of integration is fine. Shift
in the first integral $t-s\to t$, $t+s \to t$
\[ \hat{H}_f(\omega) = \frac{1}{\pi} \lim_{\epsilon \to 0} \left[
\improp \int_{\epsilon}^{\infty} \frac{1}{s} ( 
\underbrace{f(t)e^{i\omega t}}_{\hat{f}(\omega)}e^{i\omega s}
- \underbrace{f(t) e^{i\omega t}}_{\hat{f}(\omega)}e^{-i\omega s} 
)ds \, dt \right] \]
\[ = \frac{1}{\pi} \lim_{\epsilon \to 0} \hat{f}(\omega) \int_{\epsilon}^{\infty}
\frac{1}{s} ( e^{i\omega s} - e^{-i \omega s} ) ds  = \frac{i}{\pi} \hat{f}(\omega)
\improp \frac{\sin \omega s}{s} ds \]
Recall
\[P \improp \frac{\sin \omega s}{s} ds = \left\{ \begin{array}{cc}
\pi & \omega > 0 \\
0 & \omega =0 \\
-i\pi & \omega <0 \end{array} \right. \]
Therefore, for
\begin{alignat*}{3}
\omega > 0 & \quad & \hat{H}_f(\omega) &= \;\;\; i \hat{f}(\omega) \\
\omega < 0 & & \hat{H}_f(\omega) &= -i \hat{f}(\omega) \\
\omega =0 & & \hat{H}_f(\omega) &=0
\end{alignat*}
So
\[ \hat{H}_f\hat{H}_f (\omega) = -f(\omega) \quad (\omega \neq 0) \]
So we have that $H^{-1} = -H$.
Use of the Hilbrt transform is to discover the Kramers-Kronig relations.
Consider a function $f(s)$ of a complex variable $s$, analyic in the 
upper half plane, and dies off faster than $1/|s|$ in the upper half plane.
Take $t$ real. 
% Contour picture here,
\[ \int_c \frac{f(s)}{t-s} \]
Integrating along the real axis gives
\[ \underbrace{\frac{1}{\pi} P\improp \frac{f(s)}{t-s} ds}_{H_f(s)}
+\frac{1}{\pi} (-i\pi(\mbox{Res at } s=t)) =0\]
\[ \implies \frac{1}{\pi} \improp \frac{f(s)}{t-s} ds = -i f(t) \]
Separate $f(s)$ into its real and imaginary parts.
Real:
\[ \frac{1}{\pi} P \improp \frac{u(x,0)}{t-x} dx = v(t,0)\]
Imaginary:
\[ \frac{1}{\pi} P \improp \frac{v(x,0)}{t-x} dx = -u(t,0)\]
So if $u$ is known on the real axis, then $v$ can be found.
\end{document}
