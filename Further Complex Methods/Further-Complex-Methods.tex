%
%
\documentclass{article}
\usepackage{amsmath}
%\usepackage{empheq}
\usepackage{graphicx}
\usepackage{bm}
%\usepackage{framed}
\usepackage{enumerate}
\usepackage{array}
%\usepackage{pgfplots}
%\usepackage{caption}
\usepackage{amsfonts}
\usepackage[margin=3cm]{geometry}
%\usepackage{float}

\newcommand{\definition}{\textbf{Definition:}} 
\newcommand{\example}{\textbf{Example:}} 
\newcommand{\examples}{\textbf{Examples:}} 
\newcommand{\question}{\textbf{Question:}} 
\newcommand{\theorem}{\textbf{Theorem:}} 

\begin{document}

\title{Further Complex Methods}
\author{Course given by Prof. M.Perry \\
\LaTeX\  by Dominic Skinner \\
Dom-Skinner@github.com}
\maketitle
\begin{tabular}{lr}
\textbf{Books:} & ``Complex Variables,'' M.J Ablowitz \& A. Fokes (CUP)\\
& ``A Course in Modern Analysis,'' Whittaker \& Watson
\end{tabular}
\\
\\
\\
Any function of $x,y$ can be written as a function of $z , \;\bar{z}$ for
$z = x+iy$.
\\
\\
Functions of a complex variable are defined to be those functions of $x$ and
$y$ that can be written entirely in terms of $z$ only.
\\
\\
A function of a complex variable is continous if
\[ \lim_{z \to z_0} f(z) = f(z_0) \quad \mbox{(as in real analysis) } \]
The derivative of a function of a complex variable is
\[ f'(z) - \lim_{\delta z \to 0} \frac{f(z+ \delta z) - f(z)}{\delta z} \]
For a function to be differentiable, the limit must be independent of
the direction that the limit is taken.
\\
\\
If this is true, then the function is said to be differentiable at $z$.
If $f'(z)$ exists, then $f(z)$ is continous (converse not true).
\\
\\
Write $f(z) = u(x,y) + iv(x,y)$ with $u,v$ both real. Then
\[ dz \, f'(z) = \lim_{\delta z = \delta x + i \delta y \to 0} \left(
u(x+\delta x, y+ \delta y) + iv(x+\delta x,y+\delta y) - u(x,y) - iv(x,y) 
\right) \]
If $\delta y = 0$, $dz = dx$ and we get that
\[ f'(z) = u_x + i v_x \]
Suppose now that $\delta x = 0$.
\[ i \delta y\, f'(z) = u_y+iv_y \]
\[ \implies f'(z) = v_y - i u_y \]
\[ \implies v_y - iu_y = u_x + iv_x \]
\[ \implies \left. \begin{array}{cr}
v_y = u_x \\
v_x = -u_y \end{array} \right\} \mbox{ The Cauchy-Riemann equations} \]
If the Cauchy-Riemann equations hold, the derivatives exist and are 
continous, then $f(z)$ is differentiable. If the Cauchy-Rieman equations hold
then $u, \; v$ are harmonic.
\[ u_{xx} = v_{xy} = -u_{yy} \implies u_{xx}+u_{yy} =0 \]
A similar equation holds with v.
\\
% A diagram goes here
\\
Consider surfaces of $u=const$, $v=const$. These surfaces are orthogonal.
\begin{align*}
\nabla u &= (u_x,u_y) \mbox{ -i normal to } u = const \\
\nabla v &= (v_x,v_y) \mbox{ - normal to } v = const 
\end{align*}
and so
\[ \nabla u \cdot \nabla v = u_xv_x + u_yv_y =0 \mbox{ from C-R} \]
\\
\definition\ Analytic function
\\
$f(z)$ is analytic at $z_0$ if $f(z)$ is differentiable in some neighbourgood
of $z_0$. $f(z)$ is analytic in a region if a similar condition applies.
\\
\\
\examples\
\begin{enumerate}[(i)]
\item $e^z$  is analytic in the finite complex $z$-plane
\item $\bar{z}$ is analytic nowhere
\item $1/z^3$ is analytic everywhere except at $z=0$
\end{enumerate}
\noindent \definition\ Entire functions
\\
A function is entire if it is analytic in the finite complex plane
\\
\\
\examples\
\begin{enumerate}[(i)]
\item $e^z$ , this only fails to be analytic at $\infty$ 
\item $\sin z$ 
\item $z^2$ 
\end{enumerate}
~\\
\definition\ Isolated singularity
\\
A function is said to have an isolated singularity if it fails to
be analytic at a point.
\\
\\
\example\ $1/z^3$ has an isolated singularity at the origin.
\\
\\
Suppose that a function has an isolated singularity at $z=z_0$. Then it can
be expanded as a Laurent series around $z_0$.
\[ f(z) = \sum_{-\infty}^{\infty} c_n (z - z_0)^n \]
Note that this sum is over all positive and negative powers.
\\
\\
Suppose that $c_n=0$ for all $n< -N$ where $N>0$. 
\begin{itemize}
\item If $c_n = 0 \; \forall \, n >0$ then it is not singular.
\item If $c_n = 0$ for all $n<-N$ for $N>0$, then one has a pole of order
	$N$.
\end{itemize}
\noindent \example\ $1/z^3$ has a pole of order 3 at $z=0$.
\\
The coefficient $c_{-1}$ is special, it is the residue of the pole at 
$z_0$.
\\
\\
\definition\ Removable singularities
\\
Fake singularities where the building blocks of $f(z)$ have isolated
singularities, but $f(z)$ does not.
\\
\\
\example\
\[ f(z) = \frac{\sin z}{z} = \frac{1}{z} ( z - \frac{z^3}{6} + \dots) 
= 1 - \frac{z^2}{6} \]
$f(z)$ has a removable singularity at $z=0$.
\\
\\
\example\
\[ f(z) = \frac{1}{z} - \frac{1}{z + z^2} = \frac{1}{1+z} \]
so $f(z)$ has a removable singularity at the origin.
\\
\\
\definition\ Essential Singularity \\
An essential singularity is where the order of the pole of an isolated
singularity is infinite.
\\
\\
\example\ $f(z) = e^{1/z}$, $z=0$ is an isolated singularity, as a Laurent
series
\[ f(z) = \sum_{\infty}^{0} \frac{1}{(-n)!} z^n \]
Note that in this exaple, $f(z)$ is not even continous at $z=0$, its value
depends on how one approaches $z=0$.
\end{document}
