%
%
\documentclass{article}
\usepackage{amsmath}
%\usepackage{empheq}
\usepackage{graphicx}
\usepackage{bm}
\usepackage{framed}
\usepackage{enumerate}
\usepackage{array}
%\usepackage{pgfplots}
%\usepackage{caption}
\usepackage{amsfonts}
\usepackage[margin=3cm]{geometry}
%\usepackage{float}

\newcommand{\definition}{\textbf{Definition:}} 
\newcommand{\example}{\textbf{Example:}} 
\newcommand{\examples}{\textbf{Examples:}} 
\newcommand{\question}{\textbf{Question:}} 
\newcommand{\theorem}{\textbf{Theorem:}} 
\newcommand{\pder}[2] {\frac{\partial {#1}}{\partial {#2} }}%

\begin{document}

\title{Asymptotic Methods}
\author{Course given by Dr. D. Stuart \\
\LaTeX\  by Dominic Skinner \\
Dom-Skinner@github.com}
\maketitle
\noindent \textbf{Books:} Bender and Orszag, ``\emph{Advanced Mathematical methods for 
scientists and engineers}'', Chapters 3,6,10 \\
More details can be found on the Moodle course site; self-enrol into the
Asymptotic methods course.
\\
\subsection*{What we'll learn in this course}
\textbf{Examples:} 
\begin{enumerate}[1.]
\item $\displaystyle I(\lambda) = \int_{\infty}^{\infty} \exp[-\lambda \cosh u] du $
      \\ We expect that $I(\lambda) \to 0$ as $\lambda \to \infty$. But how fast?
\\
\item $\displaystyle i \hbar \pder{\psi}{t} = -\frac{\hbar^2}{2m}
      \pder{^2 \psi}{x^2} + V(x)$ with $\psi(x,t) \in \mathbb{C}$,
      $V = V(x)$. \\
      Look for a solution $\displaystyle \psi(x,t) = \exp \left[\frac{-iEt}{\hbar} 
      \right] f(x)$
      $\implies \hbar^2 f'' = 2m(V(x) -E)f$ 
      \\
      $\hbar$ is very small. So a natural problem is to try and understand
      $\displaystyle \epsilon^2 \frac{d^2y}{dx^2} = Q(x)y$ when $\epsilon \ll 1$.
      \\ The ``semi-classical limit'' or ``geometric optics''.
\\
\item Put $\hbar = 1$, $m=\frac{1}{2}$, $V=0$; specify $\psi(x,0) = \psi_0(x)$
      \\[2pt]
      Fourier transform $\displaystyle \to \psi(x,t) = \frac{1}{(4\pi i t)^{1/2}} 
      \int_{\mathbb{R}} \exp\left[ \frac{i|x-y|^2}{4t} \right] \psi_0(y) dy$.
      \\
      \\
      Question: Does $\psi(x,t)$ really approach $\psi_0(x)$ as $t\to0$?
\end{enumerate}
%
\section{Asymptotic expansions of functions}
\[ \sinh x = \frac{e^x - e^{-x}}{2} = x + \frac{x^3}{3!} + \frac{x^5}{5!} + \dots\]
say $\sinh x \sim x$ as $x \to 0$.
\\
\\
\definition\ $f \sim g$ as $x \to x_0$ is $|f(x) - g(x)| = o(g(x))$ as $x \to x_0$.
\\
\\
\example 
\[|\sinh x - x| = |\frac{x^3}{3!} + \frac{x^5}{5!} + \dots| = O(x^3) = o(x)\]
\\
($F=O(G)$ as $x \to x_0$ means $\exists C>0$ such that $|F(x)| \leq C|G(x)|$ in
some open interval $I$, with $x_0 \in I$)
\\
In fact, by remainder estimate for Taylor expansion
\[ \left| \sinh x - \sum_{0}^{N} \frac{x^{2n+1}}{(2n+1)!} \right| = \
O(x^{2n+3}) = o(x^{2n+1}) \mbox{ as } x \to 0\]
We write $\displaystyle \sinh x \sim \sum_{0}^{\infty} \frac{x^{2n+1}}{(2n+1)!}$
\\
\\
\definition\ Asymptotic sequence and asymptotic expansion.
\begin{enumerate}[(i)]
\item $\{\phi_n\}_{n=0}^{\infty}$ is an asymptotic sequence (of functions) as 
		$x \to x_0$ if $\phi_{n+1}(x) = o(\phi_n(x))$ as $x \to x_0$. 
\item A function f has asymptotic expansion w.r.t. $\{ \phi_n\}$ as $x \to x_0$ 
		written $f \sim \sum_{n=0}^{\infty} a_n \phi_n$ if 
		\[ \left| f(x) - \sum_{n=0}^N a_n \phi_n(x) \right| = o(\phi_N(x)) \mbox{ as }
		 x \to x_0 \forall N\]
		\\
		Notice the difference with Taylor expansion - an asymptotic expansion 
		need not converge as $N\to \infty$ for any $x$!
\end{enumerate}
\examples\ 
\begin{itemize}
\item $\{ \phi_n(x) = x^n \}$ as $x \to 0$, the most common sequence.
\item $\{ \phi_n(x) = x^{2n+1} \}$ as $x \to 0$
\item $\{ \phi_n(x) = e^{-n/x} \}$ as $x \to 0^+$ (i.e. $x>0$ and $x\to 0$ on right)
\end{itemize}
%
%
\begin{framed}
\noindent \begin{tabular}{cl} 
\textbf{Warning:} & $\displaystyle \sin x \sim x - \frac{x^3}{3!} -
\frac{x^5}{5!} + \dots$ as $x \to 0$. \\
& $\displaystyle \sin x + e^{-1/x} \sim x  - \frac{x^3}{3!} + \frac{x^5}{5!}
 - \dots$ as $x \to 0^{+}$. 
\end{tabular}
\end{framed}
Why?
\[ \left| \sin x + e^{-1/x} - \sum_0^N \frac{ (-1)^n x^{2n+1}}{(2n+1)!} \right| \
= \left| \sum_{n = 2N+1}^{\infty} \frac{(-1)^n x^{2n+1}}{(2n+1)!} + e^{-1/x} \right| \
=O(x^{2N+3}) = o(x^{2N+1}) \]
Moral: information is lost in asymptotic expansions!
\\
\\
However, given $f$ and asymptotic sequence, the $a_j$'s are unique, i.e.
\begin{align*}
a_0 &= \lim_{x \to x_0} \frac{f(x)}{\phi_0(x)} \\
a_1 &= \lim_{x \to x_0} \frac{f(x)-a_0 \phi_0(x)}{\phi_1(x)} \\
& \vdots
\end{align*}
\question\ Is it possible that $f(x) \sim 0$ as $x \to 0$?
\\
If $|f(x) - 0| = o(0)=0$ in some interval $I$, containing $0$, then
$f \equiv 0$ on $I$.
\\
\\
\example\ Consider $\displaystyle Ei(x) = \int_{x}^{\infty} \frac{e^{-t}}{t} dt$
as $x \to + \infty$.
\\
\\
Consider the asymptotic sequence $\phi_n(x) = 1/x^n$ as $x \to + \infty$
\[ Ei(x) = \int_{x}^{\infty} \frac{-d(e^{-t})}{t} = \left[ -\frac{e^{-t}}{t} \right]_{x}^{\infty} \
- \int_x^{\infty} \frac{e^{-t}}{t^2} dt = \frac{e^{-x}}{x} - \
\int_x^{\infty} \frac{e^{-t}}{t^2} dt\]
Claim: $Ei(x) \sim e^{-x}/x$ as $x \to + \infty$.
\[ \left| Ei(x) - \frac{e^{-x}}{x} \right| = \left| \int_x^{\infty} \frac{e^{-t}}{t^2} dt \right|
\leq \frac{1}{x^2} \int_x^{\infty} e^{-t} dt = \frac{e^{-t}}{x^2} = \
o\left(\frac{e^{-x}}{x} \right) \]
Working out the full expansion of $Ei$ with respect to
$\phi_n = e^{-x}/x^n$ gives that.
\[ Ei(x) \sim \sum_{n=0}^{\infty} \frac{(-1)^n n! e^{-x}}{x^{n+1}} \]
\underline{What do we mean?}
\begin{enumerate}[(i)]
\item $\phi_n (x) - e^{-x}/x^{n+1}$ satisfies
		$\phi_{n+1}(x) = o(\phi_n(x))$ as $x \to + \infty$.
		i.e. it forms an ``asymptotic sequence.''
\item The notation ``$\sim$'' (``asymptotic to'') means
\[ \left|Ei(x) - \sum_{n=0}^N \frac{(-1)^n n! e^{-x} }{x^{n+1}} \right| = o\left(
\frac{e^{-x}}{x^{n+1}} \right) \mbox{ as } x \to + \infty\]
\end{enumerate}
This can be proved with integration by parts:
\begin{align*}
Ei(x) &= -\int_x^{\infty} \frac{1}{t} d(e^{-t}) = e^{-x}/x + \int_x^{\infty}
\frac{1}{t^2} d(e^{-t}) \\
&= e^{-x}/x - e^{-x}/x^2 + 2 \int_x^{\infty} \frac{e^{-t}}{t^3}dt  \\
&= e^{-x} \left[ \frac{1}{x} - \frac{1}{x^2} + \frac{2!}{x^3} - \frac{3!}{x^4}
+ \dots + \frac{(-1)^n n!}{x^{n+1}} \right] + \underbrace{(-1)^{n+1} (n+1)!
\int_x^{\infty} \frac{e^{-t}}{t^{n+2}} dt}_{Rem_{n+1}(x)}
\end{align*}
Where
\[ |Rem_{n+1}(x)| \leq \frac{(n+1)!}{x^{n+2}} \int_x^{\infty} e^{-t} dt =
\frac{(n+1)! e^{-x}}{x^{n+2}} = o\left( \frac{e^{-x}}{x^{n+1}} \right) 
\mbox{ as } x \to + \infty \]
So it is an asymptotic expansion. Not convergent because 
$\displaystyle \sum (-1)^n \,n! \, y^{n+1}$
has radius of convergence 0. (In fact for fixed $y$ the terms become unbounded.)
\\
\\
Consider magnitudes of successive terms $\displaystyle 
f_n(x) =\frac{ (-1)^n\, n!\, e^{-x}}{x^{n+1}}$
\[ \left| \frac{f_{n+1}(x)}{f_n(x)}\right| = \frac{n+1}{x} \]
% A figure goes here
\underline{Optimal truncation:} Truncate the asymptotic expansion at the 
point $n=N_x$, such that the first term excluded is the smallest.
\\
\\
In our example, choose $N_x = [x] -1 = \sup \{j-1: j \leq x, \, j\in \mathbb{N} \}$
\[ \left. \begin{array}{ccc}
\left| f_{N_x+1}(x)/f_{N_x}(x)\right| &= (N_x+1)/x &\leq 1 \\
\\
\left| f_{N_x+2}(x)/f_{N_x+1}(x)\right| &= (N_x+2)/x &> 1 \end{array}
\right\} \mbox{ so } f_{N_x+1} \mbox{ is the smallest term, later terms are larger} \]
So we write
\[ Ei(x) = \sum^{N_x} \frac{(-1)^n n! e^{-x}}{x^{n+1}} + Rem_{N_x+1}(x) \]
\[ |Rem_{n+1}(x)| \leq \frac{(N_x+1)!}{x^{N_x+2}}e^{-x} =
\frac{[x]! e^{-x}}{x^{[x]+1}} \leq
\frac{2 \left( \frac{[x]}{e} \right)^{[x]} \sqrt{2\pi [x]} e^{-x}}{x^{[x]+1}}
\leq \frac{2\sqrt{2\pi[x]}}{[x]} e^{-x} e^{-[x]} \]
Where we have used Stirling's formula. 
\[ \lim_{n \to \infty} \frac{n!}{(n/e)^n \sqrt{2\pi n}} \to 1 \mbox{ as } n \to \infty \]
The good new is the additional 
$e^{-[x]}$ term. Optimal truncation (often) gives an exponentially small 
remainder.
\\
\\
\examples\
\[ \sinh x = \frac{e^x - e^{-x}}{2} \sim e^{-x}/2 \mbox{ as } x \to + \infty \]
Works because $e^{-x} = o(e^x)$ as $x \to \infty$
\\
\\
\[ \mathrm{sech}  \, x = \frac{2}{e^x + e^{-x}} = \frac{2}{e^x}(1 + e^{-2x})^{-1} =
\frac{2}{e^x}(1 - e^{-2x} + e^{-4x} - \dots) \]
This gives an asymptotic expansion for the sequence $\phi_n = e^{-nx}$
(Which is asymptotic since $e^{-x} = o(e^x)$ as $x \to + \infty$)
\\
\\
Note: $\sinh x \sim -e^{-x}/2$ as $x \to -\infty$
\\
\\
Consider $\sinh z, \mbox{ for }\, z \in \mathbb{C} $
\[ \sinh z = \frac{e^z- e^{-z}}{2} = \frac{e^{x+iy} - e^{-x-iy}}{2} \]
\begin{alignat*}{2}
& \sim  e^z/2 &\mbox{ as } z \to \infty \mbox{ in sector } \{ -\frac{\pi}{2}
< arg(z) < \frac{\pi}{2} \} \\
& \sim  e^{-z}/2 &\mbox{ as } z \to \infty \mbox{ in sector } \{ \frac{\pi}{2}
< arg(z) < \frac{3\pi}{2} \} 
\end{alignat*}
\underline{Conclusion:} The asymptotic seems to change suddenly when going from
sector to sector. % Figure of said sectors
\\
\\
The lines separating the different sectors are Stokes Lines.
\\
\\
\textbf{Excercise:} Prove that the definition of asymptotics in a sector must satisfy
that you do not approach Stokes lines too fast.

\end{document}
