%
% Catam project simulating wind forced ocean currents
%
\documentclass{article}
\usepackage{amsmath}
\usepackage{empheq}
\usepackage{graphicx}
%\usepackage{epstopdf}
%\usepackage{color}
%\usepackage{pgfplots}
\usepackage{caption}
\usepackage{amsfonts}
\usepackage[margin=3cm]{geometry}
%\usepackage{float}

\begin{document}

\title{Dynamical Systems}
\author{Dominic Skinner}
\maketitle
\section*{Introduction}
A dynamical system is a set of equations describing the evolution of a system
with respect to a time-like variable. Usually they are non-linear.
\\
The possible states of the system define the state space/phase space.
\\
\\
\textbf{Example:}   The logistic map
\[ x_{n+1} = \mu x_n ( 1- x_n) \]
For $ 0 \leq \mu \leq 4 $ this describes evolution with respect to a discrete time 
$n$ in a state space $[0,1]$.
\\
\\
\textbf{Example:}   The Lotka-Voltera equations
\begin{align*}
\dot{r} &= r(a - br -cs) \\
\dot{s} &= s(d - er -fs)
\end{align*}
Where \emph{a-f} are positive constants. These describe continous evolution 
in a state space $(r,s) \in [0 , \infty] \times [0, \infty]$ as a model for 
the population of two species competing for the same food supply.
\\
\\
\textbf{Example:}   The non-linear Schr\"odinger equation
\[ i \frac{\partial \Psi}{ \partial t} = \nabla^2 \Psi + |\Psi|^2 \Psi \]
This describes evolution in an infinite dimensional statespace of possible
wavefunctions.
\\
\\
Because the equations are non-linear, it is often impossible to find a complete
set of closed form analytic solutions. Instead, we resort to a mixture of 
geometric and analytic arguments, and aim to say something about the generic
long-term behaviour.
\\
\\
\textbf{Example:}   
\begin{align*}
\dot{r} &= r(3 - r -s) \\
\dot{s} &= s(2 - r -s)
\end{align*}
Consider the regions where $\dot{r}$ and $\dot{s}$ are $>0, \; <0, \; =0$. 
\\
If $r,s >0$ then
\begin{align*}
r+s &< 2 \implies \dot{r}, \dot{s} > 0 \\
2 < r+s &< 3 \implies \dot{r} >0 , \; \dot{s} < 0 \\
r+s &> 3 \implies \dot{r}, \dot{s} < 0 \\
\end{align*}
$\dot{r} = 0$ if $r=0$ or $r+s = 3$. \\
$\dot{s} = 0$ if $s=0$ or $r+s = 2$. \\
Therefore $\dot{r} = \dot{s} = 0$ at the fixed points $(0,0), \; (3,0), \; (0,2)$.
This gives the phase portrait/diagram/plane%
\footnote{Phase portrait, phase diagram, phase plane will be used interchangably}
% INCLUDE GRAPHIS HERE
The most important feature of the phase portrait is that all solutions with 
$r >0$ tend to the stable fixed point $(3,0)$. The fixed points $(0,0)$ and
$(0,2)$ are unstable. There are no periodic orbits.
\\
\\
\textbf{Example:}   
\begin{align*}
\dot{r} &= r(3 - r -s) \\
\dot{s} &= s(2 - \mu r -s)
\end{align*}
In this case, a new fixed point $(\frac{1}{1- \mu} , \frac{2 - 3 \mu}{1- \mu})$
appears in the state space at $\mu = \frac{2}{3}$ and for $\mu < \frac{2}{3}$ 
is the long term stable attractor. 
\\
A qualitative change in the solution structure is called a bifurcation.
\\
\\
\textbf{Example:}   
\begin{align*}
\dot{x} &= -y + \epsilon x (\mu - x^2 - y^2) \\
\dot{y} &= \;\;\; x + \epsilon y (\mu - x^2 - y^2)
\end{align*}
Use polar coordinates which are a more natural choice for this problem.
In general:
\begin{empheq}[box=\fbox]{align}
\dot{r} &= \frac{x \dot{x} + y \dot{y}}{r} \nonumber \\
\dot{\theta} &= \frac{x \dot{y} - y \dot{x} }{ r^2} \nonumber
\end{empheq}
Which are equations that will be referred to frequently. \footnote{so learn them now!}
In our example, they become
\begin{align*}
\dot{r} &= \epsilon r (\mu - r^2) \\
\dot{\theta} &= 1
\end{align*}
Consider $\dot{r}$ and $\dot{\theta}$:
% INSERT DIAGRAM HERE
\\
The infinite set of periodic solutions for $\epsilon = 0$ is destroyed by any 
perturbation to $\epsilon \neq 0$. This is an example of structural instability.
If $\mu > 0$ then just one limit cycle survives and is stable (unstable) for 
$\epsilon >0$ ($\epsilon < 0$). The appearance of the limit cycle as $\mu \uparrow$ 
through $0$ is another form of bifurcation.
\\
\\
\textbf{Example:} In 2D the points of successive interection $x_n$ of a 
solution near a limit cycle with a line $\varepsilon$ perpendicular to the cycle,
move monotonically towards/away from the point of intersection $x^{*}$ of 
$\varepsilon$ with the limit cycle.
% INSERT PLOT HERE
\\
The point $x^{*}$ is a stable/unstable fixed point of this Poincar\'e recurrence
map.
\\
In 3 or higher dimensions, or in 2D with time-dependent coefficients there is 
room for much more complicated behaviour including \underline{chaos}.
%%%%%%%%%%%%%CHAPTER 1
\section{Basic Definitions}
We need some termonology.
\subsection{Notation}
We only consider ODEs of the form 
\begin{equation}\tag{*}
\dot{\textbf{x}} = \textbf{f(x)}
\end{equation}
for \textbf{x} in a phase space/state space $E \subset \mathbb{R}^{n}$.
The n first order ODEs form a dynamical system of order (dimension) n.
\\
\\
Since
\[ \frac{\partial \textbf{f} } { \partial t} = 0 \]
 we call the system \underline{autonomous}.
\\
\\
A non-autonomous system $\dot{\textbf{x}} = \textbf{f}(\textbf{x} ,t) $ can be
made autonomous by setting
\[ \textbf{y} = ( \textbf{x},t) \quad \mbox{ with } \quad \dot{\textbf{y}} = ( \textbf{f}(\textbf{y}),1) \]
The n$^{th}$ order ODE
\[ \frac{d ^n x}{d t^n} = g \left( x , \frac{d x}{d t} , \dots , \frac{d^{n-1} x}{d t^{n-1}} \right) \]
can be put in the form (*) by setting
\[ \textbf{y} =  \left( x , \frac{d x}{d t} , \dots , \frac{d^{n-1} x}{d t^{n-1}} \right) \]
with 
\[ \dot{\textbf{y}} =  \left( y_2 , y_3 , \dots , g( \textbf{y}) \right) \]
Similarly we will consider maps in the form
\[ \textbf{x}_{n+1} = \textbf{F}(\textbf{x}_n) \]



\end{document}
